\documentclass[oneside,final,14pt]{extreport}

%% my command
%%%%%%%%%%%%%%
% Путь к файлу с изображениями
\newcommand{\picPath}{img}
% Величина отступа
\newcommand{\indentSpace}{1.25cm}
% Сокращения
\newcommand{\urlTitle}{ $-$ URL: }
%%%%%%%%%%%%%%%


% Изменяем шрифт
\usepackage{fontspec}
\setmainfont{Times New Roman}
\listfiles

% Полуторный интервал
\linespread{1.6}

% Отступ
\setlength\parindent{\indentSpace}

% Математика
\usepackage{mathtools}


% Картинки
\usepackage{graphicx}
\usepackage{subcaption}

% Языковой пакет
\usepackage[russianb]{babel}

% Таблицы
\usepackage{tabularx}

% Настройка подписей к фигурам
% Меняем заголовки картинок
\usepackage[ labelsep= endash]{caption}
\captionsetup{%
   figurename= Рисунок,
   tablename= Таблица,
   justification= centering% Формат - по центру
}         

% Кирилица в подфигурах
\renewcommand{\thesubfigure}{\asbuk{subfigure}}
% разделитель в подфигурах - правая скобка
\DeclareCaptionLabelSeparator{r_paranthesis}{)\quad }
\captionsetup[subfigure]{labelformat=simple, labelsep=r_paranthesis}

% Добавляем итератор \asbuk,
% чтобы использовать кирилицу
% как маркеры
\usepackage{enumitem}
\makeatletter
\AddEnumerateCounter{\asbuk}{\russian@alph}{щ}
\makeatother

% Меняем маркеры в перечислениях
% Списки уровня 1
\setlist[enumerate,1]{label=\arabic*),ref=\arabic*}
% Списки уровня 2
\setlist[enumerate,2]{label=\asbuk*),ref=\asbuk*}
% Перечисления
\setlist[itemize,1]{label=$-$}
% Удаляем отступы перед и после
% списка
\setlist[itemize]{noitemsep, topsep=0pt}
\setlist[enumerate]{noitemsep, topsep=0pt}

% Красная строка в начале главы
\usepackage{indentfirst}

% Убиваем перенос
\usepackage[none]{hyphenat}

% Перенос длинных ссылок
\usepackage[hyphens]{url}
\urlstyle{same}

% Выравнивание по ширине
\usepackage{microtype}

%\usepackage[fontfamily=courier]{fancyvrb}
%\usepackage{verbatim}%     configurable verbatim
% \makeatletter
%  \def\verbatim@font{\normalfont\sffamily% select the font
%                     \let\do\do@noligs
%                     \verbatim@nolig@list}
%\makeatother

% Границы
\usepackage{vmargin}
\setpapersize{A4}
% отступы
%\setmarginsrb 
%{3cm} % левый
%{2cm} % верхний
%{1cm} % Правый
%{2cm} % Нижний
%{0pt}{0mm} % Высота - отступ верхнего колонтитула
%{0pt}{0mm} % Высота - отступ нижнего  колонтитула

\setlength\hoffset{0cm}
\setlength\voffset{0cm}
\usepackage[top=2cm, bottom=2cm, left=3cm, right=2cm,
]{geometry}
 		
% Настройка заглавиий
\addto\captionsrussian{% Replace "english" with the language you use
  \renewcommand{\contentsname}% содержания
    {\hfill\bfseries
    СОДЕРЖАНИЕ
	\hfill    
    }%
   \renewcommand{\bibname}% списка источников
    {\hfill\bfseries
    	СПИСОК ИСПОЛЬЗОВАННЫХ ИСТОЧНИКОВ
	\hfill
	}% 
}%\

%\renewcommand{\contentsname}{\hfill\bfseries СОДЕРЖАНИЕ \hfill} 

% Настройка  заглавий в главах
\usepackage{titlesec}


%\titleformat
%{\chapter} % command
%[display]
%{
%\bfseries
%} % format
%{
%\thechapter.
%} 	% label
%{ 
%	0 pt
%} % sep
%{    
%\centering
%} % before-code

\titleformat{\chapter}
            {\bfseries}
            {\hspace{\indentSpace}\thechapter\hspace{1em}}
            {0pt}
            {
            \vspace{0mm} }
            [\vspace{14pt}]% Отступ после
% Начальный сдвиг заголовка 50 pt = 1.763888888cm.
% Второй параметр- сдвиг до = 2cm - 50pt
\titlespacing{\chapter}{0pt}{-0.2361cm}{0pt}

\titleformat{\section}
{\bfseries}{\hspace{\indentSpace}\thesection}{1em}{}

\titleformat{\subsection}
{\bfseries}{\hspace{\indentSpace}\thesubsection}{1em}{}

%\titleformat{\section}
%            {\bfseries}
%            {\thechapter.\hspace{1em}}
%            {0pt}
%            {\centering
%            \vspace{0mm} }
%            [\vspace{14pt}]% Отступ после
%\titlespacing{\section}{0pt}{-50pt}{0pt}

% Конец настройка заглавий

% Форматирование списка источников
\makeatletter
\renewcommand*{\@biblabel}[1]{\hfill#1}
\makeatother

% Убрать отсупы в списке источников
\usepackage{lipsum}

% ADD THE FOLLOWING COUPLE LINES INTO YOUR PREAMBLE
\let\OLDthebibliography\thebibliography
\renewcommand\thebibliography[1]{
  \OLDthebibliography{#1}
  \setlength{\parskip}{0pt}
  \setlength{\itemsep}{0pt plus 0.3ex}
}



% Добавить точки в оглавление
\usepackage{tocstyle}
\newcommand{\autodot}{.}


% Чтобы картинки вставляись
% куда надо
\usepackage{float}

% Для вычисления кол-ва страниц
\usepackage{lastpage}

% Для вычисления кол-ва рисунков и таблиц
%%%
\usepackage{etoolbox}

\newcounter{totfigures}
\newcounter{tottables}

\providecommand\totfig{} 
\providecommand\tottab{}

\makeatletter
\AtEndDocument{%
  \addtocounter{totfigures}{\value{figure}}%
  \addtocounter{tottables}{\value{table}}%
  \immediate\write\@mainaux{%
    \string\gdef\string\totfig{\number\value{totfigures}}%
    \string\gdef\string\tottab{\number\value{tottables}}%
  }%
}
\makeatother

\pretocmd{\chapter}{\addtocounter{totfigures}{\value{figure}}\setcounter{figure}{0}}{}{}
\pretocmd{\chapter}{\addtocounter{tottables}{\value{table}}\setcounter{table}{0}}{}{}
%%%

% Режим релиза
\sloppy
\usepackage{layout}

%\renewcommand{\arraystretch}{1.6}

\newcommand{\cmmnt}[1]{\ignorespaces}
\newcommand{\bs}{\boldsymbol}
\usepackage{breqn}
\begin{document}
Дано $X = {x_i}, Y = {y_i}, i = \overline{1,n}$
Длинна интервала $len$
\begin{equation}
len
=
x_{max} - x_{min}
\end{equation}
Размер интервала $h$
\begin{equation}
h
=
\frac{len}{1 +3.28ln(len)}
\end{equation}
Кол-во интервалов $m$
\begin{equation}
m
=
\begin{bmatrix}
\frac{len}{n}
\end{bmatrix}
\end{equation}
Выборочное среднее $\overline{x}$
\begin{equation}
\overline{x}
=
\frac{
\sum\limits_{i=1}^n x_i
}
{n}
\end{equation}
Выборочная дисперсия $D$
\begin{equation}
D
=
\frac{
\sum\limits_{i=1}^n  n_i(\overrightarrow{x_i} - x_i)^2
}{n}
\end{equation}
 где $\overrightarrow{x_i}$ - середина $i$ -го диапазона, а $n_i$ -  кол-во элементов в этом диапазоне
 
cреднеквадратическое отклонение $\sigma$
\begin{equation}
\sigma
=
\sqrt{D}
\end{equation}

Несмещенная оценка дисперсии $S^2$
\begin{equation}
S^2
=
\frac{n}{n-1}
D
\end{equation}

Несмещенная оценка cреднеквадратического отклонения $S$
\begin{equation}
S
=
\sqrt{\frac{n}{n-1}}
\sigma
\end{equation}

Параметр $\delta$ 
\begin{equation}
\delta
=
\frac{t\sigma}{\sqrt{n}}
\end{equation}
\begin{equation}
\Phi(t)=
\frac{
\gamma
}
{2}
\end{equation}
где $\gamma$ - критерий точности $=0.95$, $\Phi$ - функция Лапласа

Доверительный интервал оценки мат ожидания
\begin{equation}
(\overline{x} - \delta,
\overline{x} + \delta)
\end{equation}

Доверительный интервал оценки среднеквадратического отклонения
\begin{equation}
\frac{S}{1+q}
<
\sigma
<
\frac{S}{1-q}
\end{equation}

где $q$ - параметр, зависящий от  $n$ и $\gamma$

Критерий пирсона $K$
\begin{equation}
K
=
\sum\limits_{i=1}^m
\frac{(n_i-nP_i)^2}{nP_i}
\end{equation}
\begin{equation}
P_i
=
\frac{h}{s}
\varphi(u_i)
\end{equation}
$\varphi$ - плотность нормальной вероятности

\begin{equation}
\varphi(u)
=
\frac{1}{\sqrt{2\pi}}
e^{-\frac{u^2}{2}}
\end{equation}
\begin{equation}
u_i
=
\frac{
\overrightarrow{x}
-
\overline{x}
}
{S}
\end{equation}

Если значение  $K$ больше $\chi_{critic}$, то гипотеза отвергается. Иначе принимается. $\chi_{critic}(m-3,\gamma)$ - табличное значение.

Коэффициен кореляции $r$
\begin{equation}
r
=
\frac{
\sum\limits_{i=1}^n (x_i - \overline{x})
\sum\limits_{i=1}^n (y_i - \overline{y})
}{
\sqrt{
\sum\limits_{i=1}^n (x_i - \overline{x})^2
\sum\limits_{i=1}^n (y_i - \overline{y})^2
}
}
\end{equation}

Для  линейной функции регрессии $y = ax + b$ оптимальные коэффициенты вычилсяются по формуле
\begin{equation}
a
=
\frac{n
\sum\limits_{i=1}^n x_i y_i -
\sum\limits_{i=1}^n x_i
\cdot
\sum\limits_{i=1}^n y_i}{
n
\sum\limits_{i=1}^n x_i^2
-
(
\sum\limits_{i=1}^n x_i
)^2}
\end{equation}
\begin{equation}
b
=
\frac{
\sum\limits_{i=1}^n y_i
- a \cdot 
\sum\limits_{i=1}^n x_i}{n}
\end{equation}
\end{document}